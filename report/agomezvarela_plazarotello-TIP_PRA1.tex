\documentclass[12pt]{article}

% Language setting
% Replace `english' with e.g. `spanish' to change the document language
\usepackage[spanish]{babel}

% Set page size and margins
% Replace `letterpaper' with`a4paper' for UK/EU standard size
\usepackage[a4paper,top=2cm,left=3cm,right=1.5cm,marginparwidth=1.75cm]{geometry}

% Useful packages
\usepackage{amsmath}
\usepackage{xcolor}
\usepackage{graphicx}
\usepackage[font=footnotesize,labelfont=bf]{caption}
\usepackage[allbordercolors=orange]{hyperref}
\usepackage{fontspec}
\usepackage{fancyhdr}
\usepackage{titling}
\usepackage{array}
\usepackage{multicol}
\usepackage{framed}
\usepackage{enumitem}
\usepackage[bibstyle=numeric, citestyle=numeric, sorting=nyt]{biblatex}
\usepackage{soul}
\usepackage{subcaption}
\usepackage{wrapfig}
\usepackage{listings}
\usepackage{layout}

\addbibresource{TIPPRA1.bib}

\definecolor{uocblue}{HTML}{000078}
\colorlet{main}{uocblue}
\setmainfont{Arial}[Color = uocblue]
\renewcommand{\labelitemi}{$\textcolor{uocblue}{\bullet}$}
\renewcommand{\labelitemii}{$\textcolor{uocblue}{\cdot}$}
\renewcommand{\labelitemiii}{$\textcolor{uocblue}{\diamond}$}
\renewcommand{\labelitemiv}{$\textcolor{uocblue}{\ast}$}
\renewcommand{\thefootnote}{\textbf{\arabic{footnote}}}
\renewcommand{\footnoterule}{{
		\color{uocblue}
		\kern 20pt
		\hrule width 2in
		\kern 3pt
}}
\setulcolor{uocblue}

\setcounter{biburllcpenalty}{7000}
\setcounter{biburlucpenalty}{8000}

\hypersetup{allcolors=uocblue}
\urlstyle{same}

% metadata
%%%%%%%%%%%%%%
\newcommand{\master}{Máster en Ciencia de Datos}
\newcommand{\grade}{2021/2022}
\newcommand{\subject}{Tipología y ciclo de vida del dato}
\newcommand{\testname}{\textbf{PRA1}}
\newcommand{\loginuocone}{agomezvarela@uoc.edu}
\newcommand{\loginuoctwo}{plazarotello@uoc.edu}
\newcommand{\authorone}{Alba Gómez Varela}
\newcommand{\authortwo}{Patricia Lázaro Tello}
\title{\textit{Web scraping} del mercado inmobiliario}
\author{Alba Gómez Varela\\Patricia Lázaro Tello}
\date{}	% quitar la fecha del titulo
%%%%%%%%%%%%%%

\pretitle{\vspace{-2cm}\begin{center}\LARGE \bfseries}
	\posttitle{\end{center}}
\preauthor{\vspace{-0.5cm}\begin{center}
		\large \itshape
		\begin{tabular}[t]{c}}
		\postauthor{\end{tabular}\end{center}\vspace{-1.5cm}}


\fancypagestyle{uoc}{
	\fancyhf{}
	\fancyhead[C]{\includegraphics[width=16.5cm]{uoc.png}}
	\fancyhead[R]{\footnotesize \hspace*{\fill}\authorone \space \space · \space \nolinkurl{\loginuocone}\\
		\hspace*{\fill}\authortwo \space \space \space · \space\space\space \nolinkurl{\loginuoctwo}\\ \vspace{-2pt}}
	\fancyfoot[R]{\footnotesize \thepage} %\hspace{0.5cm}}
\fancyfoot[C]{\footnotesize \hspace{1.5cm} \grade}
\fancyfoot[L]{\footnotesize \subject \space · \testname\\
	\space\master}
\renewcommand{\headrulewidth}{0pt} % remove lines as well
\renewcommand{\footrulewidth}{0pt}
\setlength{\footskip}{50pt}%
\setlength{\headsep}{15pt}
\setlength{\headheight}{42pt}
\setlength{\parskip}{10pt}
\setlength{\topmargin}{-1cm}
%\setlength{\textheight}{\textheight}
\setlength{\textheight}{650pt}
\setlength{\oddsidemargin}{0pt}
\setlength{\marginparsep}{0pt}
\setlength{\marginparpush}{0pt}
\setlength{\voffset}{-10pt}
}

\pagestyle{uoc}
\linespread{1.125}
\color{uocblue}

\newlength{\leftbarwidth}
\setlength{\leftbarwidth}{0.15em}
\newlength{\leftbarsep}
\setlength{\leftbarsep}{0pt}
\renewenvironment{leftbar}{%
	\def\FrameCommand{{\vrule width \leftbarwidth\relax\hspace {\leftbarsep}}}%
	\MakeFramed {\advance \hsize -\width \FrameRestore }%
}{%
	\endMakeFramed
}

\newenvironment{blockquote}{%
	\par%
	\leftskip=2em
%	\begin{leftbar}
	\noindent\ignorespaces}{%
%	\end{leftbar}
	\par}

\setlength{\parindent}{0em}

%%%%%%%%%%%
\definecolor{codegreen}{rgb}{0,0.6,0}
\definecolor{codegray}{rgb}{0.5,0.5,0.5}
\definecolor{codepurple}{rgb}{0.58,0,0.82}
\definecolor{backcolour}{rgb}{0.97,0.97,0.95}

\lstdefinestyle{mystyle}{backgroundcolor=\color{backcolour}, commentstyle=\color{codegreen},
	keywordstyle=\color{magenta}, numberstyle=\tiny\color{codegray}, stringstyle=\color{codepurple},
	basicstyle=\ttfamily\footnotesize, breakatwhitespace=false, breaklines=true, captionpos=t,                    
	keepspaces=true, numbers=left, numbersep=5pt, showspaces=false,	showstringspaces=false,
	showtabs=false, tabsize=4}

\lstset{style=mystyle}

\renewcommand*{\lstlistingname}{Documento}
%%%%%%%%%%%

\begin{document}

%\layout*

\maketitle
\thispagestyle{uoc}

\vspace{-2em}
\begin{framed}
	\begin{multicols}{2}
		\begin{itemize}[topsep=0cm,partopsep=0cm,label={},wide]
			\item \textbf{Título}: título
			\item \textbf{Licencia}: licencia
			\item \textbf{DOI}:
			\columnbreak
			\item \textbf{Número de campos}: numero
			\item \textbf{Número de registros}: numero
			\item \textbf{Fecha de extracción}: fecha
		\end{itemize}
	\end{multicols}
\end{framed}

\vspace{-2em}

\section*{1. Contexto}\vspace{-1.5em}

Explicar en qué contexto se ha recolectado la información. Explicar
por qué el sitio web elegido proporciona dicha información

Mercado inmobiliario, nueva legislación, contexto inflación...

\section*{2. Título del dataset }\vspace{-1.5em}

\section*{3. Descripción del dataset }\vspace{-1.5em}

Desarrollar una descripción breve del conjunto de
datos que se ha extraído. Es necesario que esta descripción tenga sentido con
el título elegido.

\section*{4. Representación gráfica }\vspace{-1.5em}

Dibujar un esquema o diagrama que identifique el
dataset visualmente y el proyecto elegido

\section*{5. Contenido }\vspace{-1.5em}

Explicar los campos que incluye el dataset, el periodo de tiempo de los datos y cómo se han recogido

\section*{6. Agradecimientos }\vspace{-1.5em}

Presentar al propietario del conjunto de datos. Es necesario
incluir citas de análisis anteriores o, en caso de no haberlas, justificar esta
búsqueda con análisis similares. Justificar qué pasos se han seguido para
actuar de acuerdo a los principios éticos y legales en el contexto del proyecto

\section*{7. Inspiración }\vspace{-1.5em}

Explicar por qué es interesante este conjunto de datos y qué
preguntas se pretenden responder. Es necesario comparar con los análisis
anteriores presentados en agradecimientos

\section*{8. Licencia }\vspace{-1.5em}

Seleccionar una de estas licencias para el dataset resultante y
justificar el motivo de su selección: 
● Released Under CC0: Public Domain License.
● Released Under CC BY-NC-SA 4.0 License.
● Released Under CC BY-SA 4.0 License.
● Database released under Open Database License, individual contents
under Database Contents License.
● Other (specified above).
● Unknown License.

\section*{9. Código }\vspace{-1.5em}

El \textbf{código} para el desarrollo de este proyecto se puede consultar al completo dentro de \textit{house-scraper} en: https://github.com/plazarotello/web-scraping

\section*{10. Dataset }\vspace{-1.5em}

Blablablá

\section*{11. Vídeo }\vspace{-1.5em}

Blablablá

\section*{ Tabla de contribuciones }\vspace{-1.5em}

\begin{table}[h]
	\centering
	\begin{tabular}{ | >{\arraybackslash}p{0.5\textwidth} | >{\centering\arraybackslash}p{0.2\textwidth} |}
		\hline
		\textbf{Contribuciones} &
		\textbf{Firma} \\ \hline
		\begin{tabular}{l}Investigación previa\end{tabular} &
		\begin{tabular}[c]{@{}c@{}}AGV, PLT\end{tabular} \\ \hline
		\begin{tabular}{l}Desarrollo de código\end{tabular} &
		\begin{tabular}[c]{@{}c@{}}AGV, PLT\end{tabular} \\ \hline
		\begin{tabular}{l}Redacción de las respuestas\end{tabular} &
		\begin{tabular}[c]{@{}c@{}}AGV, PLT\end{tabular} \\ \hline
	\end{tabular}
\end{table}

\newpage
\vspace{0.3cm}
\nocite{*}
\printbibliography[title={Referencias}]

\end{document}
